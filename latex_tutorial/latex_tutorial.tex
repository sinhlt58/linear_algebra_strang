\documentclass{article}

\title{My first document}
\date{27-05-2020}
\author{SinhBlack}

\usepackage{amsmath}
\usepackage{graphicx}
\usepackage{subcaption}

\begin{document}
    \pagenumbering{gobble}
    \tableofcontents

    \maketitle
    \newpage
    \pagenumbering{arabic}

    \section{Section}
    Hello World!

    \subsection{Subsection}
    Structuring a document is easy!

    \subsubsection{Subsubsection}
    More text.

    \paragraph{Paragraph}
    Some more text.

    \subparagraph{Subparagraph}
    Even more text.

    \section{Another section}

    \begin{equation*}
        f(x) = x^2
    \end{equation*}

    This formula $x = x + 2$ is an example

    \begin{equation*}
        x + y + z = b1
    \end{equation*}

    \begin{equation*}
        y + z = b2
    \end{equation*}

    \begin{equation*}
        z = b3
    \end{equation*}

    \begin{align*}
        x + y + z &= b1\\
            y + z &= b2\\
                z &= b3
    \end{align*}

    \begin{align*}
        f(x) &= x^2\\
        g(x) &= \frac{1}{x}\\
        F(x) &= \int^a_b \frac{1}{3}x^3
    \end{align*}

    \begin{align*}
        \frac{1}{\sqrt{x}}
    \end{align*}

    \begin{align*}
        \left[
        \begin{matrix}
            1 & 0 \\
            0 & 1
        \end{matrix}
        \right]
    \end{align*}

    \begin{equation*}
        \left(
            \frac{1}{\sqrt{x}}
        \right)
    \end{equation*}

    \newpage
    \section{Figures}
    \begin{figure}[h!]
    \centering
        \includegraphics[width=70mm]{cat.jpg}
        \caption{A funny cat.}
        \label{fig:cat1}
    \end{figure}

    \begin{figure}[h!]
        \centering
        \begin{subfigure}[b]{0.4\linewidth}
            \includegraphics[width=\linewidth]{cat2.jpg}
            \caption{Funny cat 2}
        \end{subfigure}
        \begin{subfigure}[b]{0.4\linewidth}
            \includegraphics[width=\linewidth]{cat2.jpg}
            \caption{Funny cat 2}
        \end{subfigure}
    \end{figure}

    \begin{appendix}
        \listoffigures
    \end{appendix}
\end{document}
